\documentclass[11pt]{article}

    \usepackage[breakable]{tcolorbox}
    \usepackage{parskip} % Stop auto-indenting (to mimic markdown behaviour)
    

    % Basic figure setup, for now with no caption control since it's done
    % automatically by Pandoc (which extracts ![](path) syntax from Markdown).
    \usepackage{graphicx}
    % Maintain compatibility with old templates. Remove in nbconvert 6.0
    \let\Oldincludegraphics\includegraphics
    % Ensure that by default, figures have no caption (until we provide a
    % proper Figure object with a Caption API and a way to capture that
    % in the conversion process - todo).
    \usepackage{caption}
    \DeclareCaptionFormat{nocaption}{}
    \captionsetup{format=nocaption,aboveskip=0pt,belowskip=0pt}

    \usepackage{float}
    \floatplacement{figure}{H} % forces figures to be placed at the correct location
    \usepackage{xcolor} % Allow colors to be defined
    \usepackage{enumerate} % Needed for markdown enumerations to work
    \usepackage{geometry} % Used to adjust the document margins
    \usepackage{amsmath} % Equations
    \usepackage{amssymb} % Equations
    \usepackage{textcomp} % defines textquotesingle
    % Hack from http://tex.stackexchange.com/a/47451/13684:
    \AtBeginDocument{%
        \def\PYZsq{\textquotesingle}% Upright quotes in Pygmentized code
    }
    \usepackage{upquote} % Upright quotes for verbatim code
    \usepackage{eurosym} % defines \euro

    \usepackage{iftex}
    \ifPDFTeX
        \usepackage[T1]{fontenc}
        \IfFileExists{alphabeta.sty}{
              \usepackage{alphabeta}
          }{
              \usepackage[mathletters]{ucs}
              \usepackage[utf8x]{inputenc}
          }
    \else
        \usepackage{fontspec}
        \usepackage{unicode-math}
    \fi

    \usepackage{fancyvrb} % verbatim replacement that allows latex
    \usepackage{grffile} % extends the file name processing of package graphics
                         % to support a larger range
    \makeatletter % fix for old versions of grffile with XeLaTeX
    \@ifpackagelater{grffile}{2019/11/01}
    {
      % Do nothing on new versions
    }
    {
      \def\Gread@@xetex#1{%
        \IfFileExists{"\Gin@base".bb}%
        {\Gread@eps{\Gin@base.bb}}%
        {\Gread@@xetex@aux#1}%
      }
    }
    \makeatother
    \usepackage[Export]{adjustbox} % Used to constrain images to a maximum size
    \adjustboxset{max size={0.9\linewidth}{0.9\paperheight}}

    % The hyperref package gives us a pdf with properly built
    % internal navigation ('pdf bookmarks' for the table of contents,
    % internal cross-reference links, web links for URLs, etc.)
    \usepackage{hyperref}
    % The default LaTeX title has an obnoxious amount of whitespace. By default,
    % titling removes some of it. It also provides customization options.
    \usepackage{titling}
    \usepackage{longtable} % longtable support required by pandoc >1.10
    \usepackage{booktabs}  % table support for pandoc > 1.12.2
    \usepackage{array}     % table support for pandoc >= 2.11.3
    \usepackage{calc}      % table minipage width calculation for pandoc >= 2.11.1
    \usepackage[inline]{enumitem} % IRkernel/repr support (it uses the enumerate* environment)
    \usepackage[normalem]{ulem} % ulem is needed to support strikethroughs (\sout)
                                % normalem makes italics be italics, not underlines
    \usepackage{mathrsfs}
    

    
    % Colors for the hyperref package
    \definecolor{urlcolor}{rgb}{0,.145,.698}
    \definecolor{linkcolor}{rgb}{.71,0.21,0.01}
    \definecolor{citecolor}{rgb}{.12,.54,.11}

    % ANSI colors
    \definecolor{ansi-black}{HTML}{3E424D}
    \definecolor{ansi-black-intense}{HTML}{282C36}
    \definecolor{ansi-red}{HTML}{E75C58}
    \definecolor{ansi-red-intense}{HTML}{B22B31}
    \definecolor{ansi-green}{HTML}{00A250}
    \definecolor{ansi-green-intense}{HTML}{007427}
    \definecolor{ansi-yellow}{HTML}{DDB62B}
    \definecolor{ansi-yellow-intense}{HTML}{B27D12}
    \definecolor{ansi-blue}{HTML}{208FFB}
    \definecolor{ansi-blue-intense}{HTML}{0065CA}
    \definecolor{ansi-magenta}{HTML}{D160C4}
    \definecolor{ansi-magenta-intense}{HTML}{A03196}
    \definecolor{ansi-cyan}{HTML}{60C6C8}
    \definecolor{ansi-cyan-intense}{HTML}{258F8F}
    \definecolor{ansi-white}{HTML}{C5C1B4}
    \definecolor{ansi-white-intense}{HTML}{A1A6B2}
    \definecolor{ansi-default-inverse-fg}{HTML}{FFFFFF}
    \definecolor{ansi-default-inverse-bg}{HTML}{000000}

    % common color for the border for error outputs.
    \definecolor{outerrorbackground}{HTML}{FFDFDF}

    % commands and environments needed by pandoc snippets
    % extracted from the output of `pandoc -s`
    \providecommand{\tightlist}{%
      \setlength{\itemsep}{0pt}\setlength{\parskip}{0pt}}
    \DefineVerbatimEnvironment{Highlighting}{Verbatim}{commandchars=\\\{\}}
    % Add ',fontsize=\small' for more characters per line
    \newenvironment{Shaded}{}{}
    \newcommand{\KeywordTok}[1]{\textcolor[rgb]{0.00,0.44,0.13}{\textbf{{#1}}}}
    \newcommand{\DataTypeTok}[1]{\textcolor[rgb]{0.56,0.13,0.00}{{#1}}}
    \newcommand{\DecValTok}[1]{\textcolor[rgb]{0.25,0.63,0.44}{{#1}}}
    \newcommand{\BaseNTok}[1]{\textcolor[rgb]{0.25,0.63,0.44}{{#1}}}
    \newcommand{\FloatTok}[1]{\textcolor[rgb]{0.25,0.63,0.44}{{#1}}}
    \newcommand{\CharTok}[1]{\textcolor[rgb]{0.25,0.44,0.63}{{#1}}}
    \newcommand{\StringTok}[1]{\textcolor[rgb]{0.25,0.44,0.63}{{#1}}}
    \newcommand{\CommentTok}[1]{\textcolor[rgb]{0.38,0.63,0.69}{\textit{{#1}}}}
    \newcommand{\OtherTok}[1]{\textcolor[rgb]{0.00,0.44,0.13}{{#1}}}
    \newcommand{\AlertTok}[1]{\textcolor[rgb]{1.00,0.00,0.00}{\textbf{{#1}}}}
    \newcommand{\FunctionTok}[1]{\textcolor[rgb]{0.02,0.16,0.49}{{#1}}}
    \newcommand{\RegionMarkerTok}[1]{{#1}}
    \newcommand{\ErrorTok}[1]{\textcolor[rgb]{1.00,0.00,0.00}{\textbf{{#1}}}}
    \newcommand{\NormalTok}[1]{{#1}}

    % Additional commands for more recent versions of Pandoc
    \newcommand{\ConstantTok}[1]{\textcolor[rgb]{0.53,0.00,0.00}{{#1}}}
    \newcommand{\SpecialCharTok}[1]{\textcolor[rgb]{0.25,0.44,0.63}{{#1}}}
    \newcommand{\VerbatimStringTok}[1]{\textcolor[rgb]{0.25,0.44,0.63}{{#1}}}
    \newcommand{\SpecialStringTok}[1]{\textcolor[rgb]{0.73,0.40,0.53}{{#1}}}
    \newcommand{\ImportTok}[1]{{#1}}
    \newcommand{\DocumentationTok}[1]{\textcolor[rgb]{0.73,0.13,0.13}{\textit{{#1}}}}
    \newcommand{\AnnotationTok}[1]{\textcolor[rgb]{0.38,0.63,0.69}{\textbf{\textit{{#1}}}}}
    \newcommand{\CommentVarTok}[1]{\textcolor[rgb]{0.38,0.63,0.69}{\textbf{\textit{{#1}}}}}
    \newcommand{\VariableTok}[1]{\textcolor[rgb]{0.10,0.09,0.49}{{#1}}}
    \newcommand{\ControlFlowTok}[1]{\textcolor[rgb]{0.00,0.44,0.13}{\textbf{{#1}}}}
    \newcommand{\OperatorTok}[1]{\textcolor[rgb]{0.40,0.40,0.40}{{#1}}}
    \newcommand{\BuiltInTok}[1]{{#1}}
    \newcommand{\ExtensionTok}[1]{{#1}}
    \newcommand{\PreprocessorTok}[1]{\textcolor[rgb]{0.74,0.48,0.00}{{#1}}}
    \newcommand{\AttributeTok}[1]{\textcolor[rgb]{0.49,0.56,0.16}{{#1}}}
    \newcommand{\InformationTok}[1]{\textcolor[rgb]{0.38,0.63,0.69}{\textbf{\textit{{#1}}}}}
    \newcommand{\WarningTok}[1]{\textcolor[rgb]{0.38,0.63,0.69}{\textbf{\textit{{#1}}}}}


    % Define a nice break command that doesn't care if a line doesn't already
    % exist.
    \def\br{\hspace*{\fill} \\* }
    % Math Jax compatibility definitions
    \def\gt{>}
    \def\lt{<}
    \let\Oldtex\TeX
    \let\Oldlatex\LaTeX
    \renewcommand{\TeX}{\textrm{\Oldtex}}
    \renewcommand{\LaTeX}{\textrm{\Oldlatex}}
    % Document parameters
    % Document title
    \title{Module 3 Assignment Report}
    \author{Bach Van Hoang Bao}
    \date{\today}
    
    
    
    
    
% Pygments definitions
\makeatletter
\def\PY@reset{\let\PY@it=\relax \let\PY@bf=\relax%
    \let\PY@ul=\relax \let\PY@tc=\relax%
    \let\PY@bc=\relax \let\PY@ff=\relax}
\def\PY@tok#1{\csname PY@tok@#1\endcsname}
\def\PY@toks#1+{\ifx\relax#1\empty\else%
    \PY@tok{#1}\expandafter\PY@toks\fi}
\def\PY@do#1{\PY@bc{\PY@tc{\PY@ul{%
    \PY@it{\PY@bf{\PY@ff{#1}}}}}}}
\def\PY#1#2{\PY@reset\PY@toks#1+\relax+\PY@do{#2}}

\@namedef{PY@tok@w}{\def\PY@tc##1{\textcolor[rgb]{0.73,0.73,0.73}{##1}}}
\@namedef{PY@tok@c}{\let\PY@it=\textit\def\PY@tc##1{\textcolor[rgb]{0.24,0.48,0.48}{##1}}}
\@namedef{PY@tok@cp}{\def\PY@tc##1{\textcolor[rgb]{0.61,0.40,0.00}{##1}}}
\@namedef{PY@tok@k}{\let\PY@bf=\textbf\def\PY@tc##1{\textcolor[rgb]{0.00,0.50,0.00}{##1}}}
\@namedef{PY@tok@kp}{\def\PY@tc##1{\textcolor[rgb]{0.00,0.50,0.00}{##1}}}
\@namedef{PY@tok@kt}{\def\PY@tc##1{\textcolor[rgb]{0.69,0.00,0.25}{##1}}}
\@namedef{PY@tok@o}{\def\PY@tc##1{\textcolor[rgb]{0.40,0.40,0.40}{##1}}}
\@namedef{PY@tok@ow}{\let\PY@bf=\textbf\def\PY@tc##1{\textcolor[rgb]{0.67,0.13,1.00}{##1}}}
\@namedef{PY@tok@nb}{\def\PY@tc##1{\textcolor[rgb]{0.00,0.50,0.00}{##1}}}
\@namedef{PY@tok@nf}{\def\PY@tc##1{\textcolor[rgb]{0.00,0.00,1.00}{##1}}}
\@namedef{PY@tok@nc}{\let\PY@bf=\textbf\def\PY@tc##1{\textcolor[rgb]{0.00,0.00,1.00}{##1}}}
\@namedef{PY@tok@nn}{\let\PY@bf=\textbf\def\PY@tc##1{\textcolor[rgb]{0.00,0.00,1.00}{##1}}}
\@namedef{PY@tok@ne}{\let\PY@bf=\textbf\def\PY@tc##1{\textcolor[rgb]{0.80,0.25,0.22}{##1}}}
\@namedef{PY@tok@nv}{\def\PY@tc##1{\textcolor[rgb]{0.10,0.09,0.49}{##1}}}
\@namedef{PY@tok@no}{\def\PY@tc##1{\textcolor[rgb]{0.53,0.00,0.00}{##1}}}
\@namedef{PY@tok@nl}{\def\PY@tc##1{\textcolor[rgb]{0.46,0.46,0.00}{##1}}}
\@namedef{PY@tok@ni}{\let\PY@bf=\textbf\def\PY@tc##1{\textcolor[rgb]{0.44,0.44,0.44}{##1}}}
\@namedef{PY@tok@na}{\def\PY@tc##1{\textcolor[rgb]{0.41,0.47,0.13}{##1}}}
\@namedef{PY@tok@nt}{\let\PY@bf=\textbf\def\PY@tc##1{\textcolor[rgb]{0.00,0.50,0.00}{##1}}}
\@namedef{PY@tok@nd}{\def\PY@tc##1{\textcolor[rgb]{0.67,0.13,1.00}{##1}}}
\@namedef{PY@tok@s}{\def\PY@tc##1{\textcolor[rgb]{0.73,0.13,0.13}{##1}}}
\@namedef{PY@tok@sd}{\let\PY@it=\textit\def\PY@tc##1{\textcolor[rgb]{0.73,0.13,0.13}{##1}}}
\@namedef{PY@tok@si}{\let\PY@bf=\textbf\def\PY@tc##1{\textcolor[rgb]{0.64,0.35,0.47}{##1}}}
\@namedef{PY@tok@se}{\let\PY@bf=\textbf\def\PY@tc##1{\textcolor[rgb]{0.67,0.36,0.12}{##1}}}
\@namedef{PY@tok@sr}{\def\PY@tc##1{\textcolor[rgb]{0.64,0.35,0.47}{##1}}}
\@namedef{PY@tok@ss}{\def\PY@tc##1{\textcolor[rgb]{0.10,0.09,0.49}{##1}}}
\@namedef{PY@tok@sx}{\def\PY@tc##1{\textcolor[rgb]{0.00,0.50,0.00}{##1}}}
\@namedef{PY@tok@m}{\def\PY@tc##1{\textcolor[rgb]{0.40,0.40,0.40}{##1}}}
\@namedef{PY@tok@gh}{\let\PY@bf=\textbf\def\PY@tc##1{\textcolor[rgb]{0.00,0.00,0.50}{##1}}}
\@namedef{PY@tok@gu}{\let\PY@bf=\textbf\def\PY@tc##1{\textcolor[rgb]{0.50,0.00,0.50}{##1}}}
\@namedef{PY@tok@gd}{\def\PY@tc##1{\textcolor[rgb]{0.63,0.00,0.00}{##1}}}
\@namedef{PY@tok@gi}{\def\PY@tc##1{\textcolor[rgb]{0.00,0.52,0.00}{##1}}}
\@namedef{PY@tok@gr}{\def\PY@tc##1{\textcolor[rgb]{0.89,0.00,0.00}{##1}}}
\@namedef{PY@tok@ge}{\let\PY@it=\textit}
\@namedef{PY@tok@gs}{\let\PY@bf=\textbf}
\@namedef{PY@tok@gp}{\let\PY@bf=\textbf\def\PY@tc##1{\textcolor[rgb]{0.00,0.00,0.50}{##1}}}
\@namedef{PY@tok@go}{\def\PY@tc##1{\textcolor[rgb]{0.44,0.44,0.44}{##1}}}
\@namedef{PY@tok@gt}{\def\PY@tc##1{\textcolor[rgb]{0.00,0.27,0.87}{##1}}}
\@namedef{PY@tok@err}{\def\PY@bc##1{{\setlength{\fboxsep}{\string -\fboxrule}\fcolorbox[rgb]{1.00,0.00,0.00}{1,1,1}{\strut ##1}}}}
\@namedef{PY@tok@kc}{\let\PY@bf=\textbf\def\PY@tc##1{\textcolor[rgb]{0.00,0.50,0.00}{##1}}}
\@namedef{PY@tok@kd}{\let\PY@bf=\textbf\def\PY@tc##1{\textcolor[rgb]{0.00,0.50,0.00}{##1}}}
\@namedef{PY@tok@kn}{\let\PY@bf=\textbf\def\PY@tc##1{\textcolor[rgb]{0.00,0.50,0.00}{##1}}}
\@namedef{PY@tok@kr}{\let\PY@bf=\textbf\def\PY@tc##1{\textcolor[rgb]{0.00,0.50,0.00}{##1}}}
\@namedef{PY@tok@bp}{\def\PY@tc##1{\textcolor[rgb]{0.00,0.50,0.00}{##1}}}
\@namedef{PY@tok@fm}{\def\PY@tc##1{\textcolor[rgb]{0.00,0.00,1.00}{##1}}}
\@namedef{PY@tok@vc}{\def\PY@tc##1{\textcolor[rgb]{0.10,0.09,0.49}{##1}}}
\@namedef{PY@tok@vg}{\def\PY@tc##1{\textcolor[rgb]{0.10,0.09,0.49}{##1}}}
\@namedef{PY@tok@vi}{\def\PY@tc##1{\textcolor[rgb]{0.10,0.09,0.49}{##1}}}
\@namedef{PY@tok@vm}{\def\PY@tc##1{\textcolor[rgb]{0.10,0.09,0.49}{##1}}}
\@namedef{PY@tok@sa}{\def\PY@tc##1{\textcolor[rgb]{0.73,0.13,0.13}{##1}}}
\@namedef{PY@tok@sb}{\def\PY@tc##1{\textcolor[rgb]{0.73,0.13,0.13}{##1}}}
\@namedef{PY@tok@sc}{\def\PY@tc##1{\textcolor[rgb]{0.73,0.13,0.13}{##1}}}
\@namedef{PY@tok@dl}{\def\PY@tc##1{\textcolor[rgb]{0.73,0.13,0.13}{##1}}}
\@namedef{PY@tok@s2}{\def\PY@tc##1{\textcolor[rgb]{0.73,0.13,0.13}{##1}}}
\@namedef{PY@tok@sh}{\def\PY@tc##1{\textcolor[rgb]{0.73,0.13,0.13}{##1}}}
\@namedef{PY@tok@s1}{\def\PY@tc##1{\textcolor[rgb]{0.73,0.13,0.13}{##1}}}
\@namedef{PY@tok@mb}{\def\PY@tc##1{\textcolor[rgb]{0.40,0.40,0.40}{##1}}}
\@namedef{PY@tok@mf}{\def\PY@tc##1{\textcolor[rgb]{0.40,0.40,0.40}{##1}}}
\@namedef{PY@tok@mh}{\def\PY@tc##1{\textcolor[rgb]{0.40,0.40,0.40}{##1}}}
\@namedef{PY@tok@mi}{\def\PY@tc##1{\textcolor[rgb]{0.40,0.40,0.40}{##1}}}
\@namedef{PY@tok@il}{\def\PY@tc##1{\textcolor[rgb]{0.40,0.40,0.40}{##1}}}
\@namedef{PY@tok@mo}{\def\PY@tc##1{\textcolor[rgb]{0.40,0.40,0.40}{##1}}}
\@namedef{PY@tok@ch}{\let\PY@it=\textit\def\PY@tc##1{\textcolor[rgb]{0.24,0.48,0.48}{##1}}}
\@namedef{PY@tok@cm}{\let\PY@it=\textit\def\PY@tc##1{\textcolor[rgb]{0.24,0.48,0.48}{##1}}}
\@namedef{PY@tok@cpf}{\let\PY@it=\textit\def\PY@tc##1{\textcolor[rgb]{0.24,0.48,0.48}{##1}}}
\@namedef{PY@tok@c1}{\let\PY@it=\textit\def\PY@tc##1{\textcolor[rgb]{0.24,0.48,0.48}{##1}}}
\@namedef{PY@tok@cs}{\let\PY@it=\textit\def\PY@tc##1{\textcolor[rgb]{0.24,0.48,0.48}{##1}}}

\def\PYZbs{\char`\\}
\def\PYZus{\char`\_}
\def\PYZob{\char`\{}
\def\PYZcb{\char`\}}
\def\PYZca{\char`\^}
\def\PYZam{\char`\&}
\def\PYZlt{\char`\<}
\def\PYZgt{\char`\>}
\def\PYZsh{\char`\#}
\def\PYZpc{\char`\%}
\def\PYZdl{\char`\$}
\def\PYZhy{\char`\-}
\def\PYZsq{\char`\'}
\def\PYZdq{\char`\"}
\def\PYZti{\char`\~}
% for compatibility with earlier versions
\def\PYZat{@}
\def\PYZlb{[}
\def\PYZrb{]}
\makeatother


    % For linebreaks inside Verbatim environment from package fancyvrb.
    \makeatletter
        \newbox\Wrappedcontinuationbox
        \newbox\Wrappedvisiblespacebox
        \newcommand*\Wrappedvisiblespace {\textcolor{red}{\textvisiblespace}}
        \newcommand*\Wrappedcontinuationsymbol {\textcolor{red}{\llap{\tiny$\m@th\hookrightarrow$}}}
        \newcommand*\Wrappedcontinuationindent {3ex }
        \newcommand*\Wrappedafterbreak {\kern\Wrappedcontinuationindent\copy\Wrappedcontinuationbox}
        % Take advantage of the already applied Pygments mark-up to insert
        % potential linebreaks for TeX processing.
        %        {, <, #, %, $, ' and ": go to next line.
        %        _, }, ^, &, >, - and ~: stay at end of broken line.
        % Use of \textquotesingle for straight quote.
        \newcommand*\Wrappedbreaksatspecials {%
            \def\PYGZus{\discretionary{\char`\_}{\Wrappedafterbreak}{\char`\_}}%
            \def\PYGZob{\discretionary{}{\Wrappedafterbreak\char`\{}{\char`\{}}%
            \def\PYGZcb{\discretionary{\char`\}}{\Wrappedafterbreak}{\char`\}}}%
            \def\PYGZca{\discretionary{\char`\^}{\Wrappedafterbreak}{\char`\^}}%
            \def\PYGZam{\discretionary{\char`\&}{\Wrappedafterbreak}{\char`\&}}%
            \def\PYGZlt{\discretionary{}{\Wrappedafterbreak\char`\<}{\char`\<}}%
            \def\PYGZgt{\discretionary{\char`\>}{\Wrappedafterbreak}{\char`\>}}%
            \def\PYGZsh{\discretionary{}{\Wrappedafterbreak\char`\#}{\char`\#}}%
            \def\PYGZpc{\discretionary{}{\Wrappedafterbreak\char`\%}{\char`\%}}%
            \def\PYGZdl{\discretionary{}{\Wrappedafterbreak\char`\$}{\char`\$}}%
            \def\PYGZhy{\discretionary{\char`\-}{\Wrappedafterbreak}{\char`\-}}%
            \def\PYGZsq{\discretionary{}{\Wrappedafterbreak\textquotesingle}{\textquotesingle}}%
            \def\PYGZdq{\discretionary{}{\Wrappedafterbreak\char`\"}{\char`\"}}%
            \def\PYGZti{\discretionary{\char`\~}{\Wrappedafterbreak}{\char`\~}}%
        }
        % Some characters . , ; ? ! / are not pygmentized.
        % This macro makes them "active" and they will insert potential linebreaks
        \newcommand*\Wrappedbreaksatpunct {%
            \lccode`\~`\.\lowercase{\def~}{\discretionary{\hbox{\char`\.}}{\Wrappedafterbreak}{\hbox{\char`\.}}}%
            \lccode`\~`\,\lowercase{\def~}{\discretionary{\hbox{\char`\,}}{\Wrappedafterbreak}{\hbox{\char`\,}}}%
            \lccode`\~`\;\lowercase{\def~}{\discretionary{\hbox{\char`\;}}{\Wrappedafterbreak}{\hbox{\char`\;}}}%
            \lccode`\~`\:\lowercase{\def~}{\discretionary{\hbox{\char`\:}}{\Wrappedafterbreak}{\hbox{\char`\:}}}%
            \lccode`\~`\?\lowercase{\def~}{\discretionary{\hbox{\char`\?}}{\Wrappedafterbreak}{\hbox{\char`\?}}}%
            \lccode`\~`\!\lowercase{\def~}{\discretionary{\hbox{\char`\!}}{\Wrappedafterbreak}{\hbox{\char`\!}}}%
            \lccode`\~`\/\lowercase{\def~}{\discretionary{\hbox{\char`\/}}{\Wrappedafterbreak}{\hbox{\char`\/}}}%
            \catcode`\.\active
            \catcode`\,\active
            \catcode`\;\active
            \catcode`\:\active
            \catcode`\?\active
            \catcode`\!\active
            \catcode`\/\active
            \lccode`\~`\~
        }
    \makeatother

    \let\OriginalVerbatim=\Verbatim
    \makeatletter
    \renewcommand{\Verbatim}[1][1]{%
        %\parskip\z@skip
        \sbox\Wrappedcontinuationbox {\Wrappedcontinuationsymbol}%
        \sbox\Wrappedvisiblespacebox {\FV@SetupFont\Wrappedvisiblespace}%
        \def\FancyVerbFormatLine ##1{\hsize\linewidth
            \vtop{\raggedright\hyphenpenalty\z@\exhyphenpenalty\z@
                \doublehyphendemerits\z@\finalhyphendemerits\z@
                \strut ##1\strut}%
        }%
        % If the linebreak is at a space, the latter will be displayed as visible
        % space at end of first line, and a continuation symbol starts next line.
        % Stretch/shrink are however usually zero for typewriter font.
        \def\FV@Space {%
            \nobreak\hskip\z@ plus\fontdimen3\font minus\fontdimen4\font
            \discretionary{\copy\Wrappedvisiblespacebox}{\Wrappedafterbreak}
            {\kern\fontdimen2\font}%
        }%

        % Allow breaks at special characters using \PYG... macros.
        \Wrappedbreaksatspecials
        % Breaks at punctuation characters . , ; ? ! and / need catcode=\active
        \OriginalVerbatim[#1,codes*=\Wrappedbreaksatpunct]%
    }
    \makeatother

    % Exact colors from NB
    \definecolor{incolor}{HTML}{303F9F}
    \definecolor{outcolor}{HTML}{D84315}
    \definecolor{cellborder}{HTML}{CFCFCF}
    \definecolor{cellbackground}{HTML}{F7F7F7}

    % prompt
    \makeatletter
    \newcommand{\boxspacing}{\kern\kvtcb@left@rule\kern\kvtcb@boxsep}
    \makeatother
    \newcommand{\prompt}[4]{
        {\ttfamily\llap{{\color{#2}[#3]:\hspace{3pt}#4}}\vspace{-\baselineskip}}
    }
    

    
    % Prevent overflowing lines due to hard-to-break entities
    \sloppy
    % Setup hyperref package
    \hypersetup{
      breaklinks=true,  % so long urls are correctly broken across lines
      colorlinks=true,
      urlcolor=urlcolor,
      linkcolor=linkcolor,
      citecolor=citecolor,
      }
    % Slightly bigger margins than the latex defaults
    
    \geometry{verbose,tmargin=1in,bmargin=1in,lmargin=1in,rmargin=1in}
    
    

\begin{document}
    
    \maketitle
    
    

    
    \section{Outline of the financial problem and numerical
procedure}\label{outline-of-the-financial-problem-and-numerical-procedure}

    \subsection{Financial Problem}\label{financial-problem}

    Given the expected value of the discounted payoff under the risk-neutral
density \(\mathbb{Q}\)

\[
V(S,t) = e^{-r(T-t)}\mathbb{E}^\mathbb{Q}[\mathbf{Payoff}(S_{T})]
\]

    Initial example data:

    \[
\begin{align*}
\text{Today's stock price }S_{0} &= 100
\\ 
\text{Strike E} &= 100
\\
\text{Time to expiry }(T - t) &= \text{1 year}
\\
\text{volatility} \sigma &= 20 \%
\\
\text{constant risk-free interest rate }r &= 5 \%
\end{align*}
\]

    \begin{tcolorbox}[breakable, size=fbox, boxrule=1pt, pad at break*=1mm,colback=cellbackground, colframe=cellborder]
\prompt{In}{incolor}{1}{\boxspacing}
\begin{Verbatim}[commandchars=\\\{\}]
\PY{c+c1}{\PYZsh{} Importing libraries}
\PY{k+kn}{import} \PY{n+nn}{pandas} \PY{k}{as} \PY{n+nn}{pd}
\PY{k+kn}{import} \PY{n+nn}{numpy} \PY{k}{as} \PY{n+nn}{np}
\PY{k+kn}{import} \PY{n+nn}{matplotlib}\PY{n+nn}{.}\PY{n+nn}{pyplot} \PY{k}{as} \PY{n+nn}{plt}
\end{Verbatim}
\end{tcolorbox}

    \begin{tcolorbox}[breakable, size=fbox, boxrule=1pt, pad at break*=1mm,colback=cellbackground, colframe=cellborder]
\prompt{In}{incolor}{2}{\boxspacing}
\begin{Verbatim}[commandchars=\\\{\}]
\PY{c+c1}{\PYZsh{} Define parameters and variables}
\PY{n}{S0} \PY{o}{=} \PY{l+m+mi}{100}
\PY{n}{E} \PY{o}{=} \PY{l+m+mi}{100}
\PY{n}{T} \PY{o}{=} \PY{l+m+mi}{1}
\PY{n}{vol} \PY{o}{=} \PY{l+m+mf}{0.2}
\PY{n}{risk\PYZus{}free\PYZus{}rate} \PY{o}{=} \PY{l+m+mf}{0.05}
\end{Verbatim}
\end{tcolorbox}

    In this report, I will use the Monte Carlo method to simulate risk
neutral random walk and then value the option under the risk neutral
framework.

    The pricing algorithm:

\begin{enumerate}
\def\labelenumi{\arabic{enumi}.}
\item
  Simulate the risk-neutral random walk starting at today's value of the
  asset over the required time horizon. This gives one realization of
  the underlying price path.
\item
  For this realization calculate the option payoff.
\item
  Perform many more such realizations over the time horizon.
\item
  Calculate the average payoff over all realizations.
\item
  Take the present value of this average; this is the option value.
\end{enumerate}

    \subsection{Simulating path using Euler - Maruyama
Scheme}\label{simulating-path-using-euler---maruyama-scheme}

    A geometric Brownian motion with a stochastic differential equation
(SDE) is given as: \[
dS = r S \: dt + \sigma S \: dW
\]

    Where:

\begin{itemize}
\item
  \(S\) is the price of the underlying
\item
  \(\sigma\) is constant volatility
\item
  \(r\) is the constant risk-free interest rate
\item
  \(X\) is the brownian motion.
\end{itemize}

    Consider \(V(S) = \log S\)

    First order derivative:

\[
\frac{dV}{dS}= \frac{1}{S} \tag{1}
\]

Second order derivative:

\[
\frac{d^{2}V}{dS^{2}} = \frac{-1}{S^{2}} \tag{2}
\]

    Using this result for \(V = V(S,t)\) \hyperref[1]{[1]}:

\[
\begin{align*}\\
 dV &= \left(r S \frac{dV}{dS} + \frac{1}{2} \sigma^{2} S^{2} \frac{d^{2}V}{dS^{2}} \right)dt     + \left(\sigma S \frac{dV}{dS} \right)dW
\end{align*} \tag{3}
\]

Subtitude (1) and (2) into (3) we have:

\[
\begin{align*}
d(\log S) &= \left(rS\left(\frac{1}{S}\right) + \frac{1}{2} \sigma^2 S^2 \left(-\frac{1}{S^2}\right) \right)dt + \sigma S \left(\frac{1}{S}\right) dW
\\\\
&= \left(r-\frac{1}{2} \sigma^{2}\right) d t+\sigma d W
\end{align*}
\]

Integrating both sides between 0 and \(t\)

\[
\begin{align*}
\int_{0}^{t} d(\log S) &= \int_{0}^{t} \left(r - \frac{1}{2} \sigma^{2}\right) d\tau + \int_{0}^{t} \sigma dW \quad (t > 0) 
\\\\
\log \frac{S_t}{S_0} &= \left(r - \frac{1}{2} \sigma^{2}\right) t + \sigma(W(t) - W(0))
\end{align*}
\]

Assuming \(W (0) = 0\) and \(S (0) = S_{0}\), the exact solution
becomes:

\[
S(t)=S_{0} \exp \left(\left(r-\frac{1}{2} \sigma^{2}\right) t+\sigma W(t)\right) \tag{4}
\]

    Using Euler - Maruyama scheme \hyperref[2]{[2]}:

\[
d S = rS \: dt + \sigma S \sqrt{dt} \: \phi
\]

where \(\phi\) is from a standardized Normal distribution. This method
has an error of \(O(\delta t)\).

Using (4) we can derive the Euler - Maruyama scheme for our problem:

\[
S_{t+ dt} = S_t \exp \left(\left(r-\frac{1}{2} \sigma^{2}\right) dt+\sigma \sqrt{d t} \: \phi \right) \tag{5}
\]

    For the random number generator for generating standard normal variable
\(\phi\), I will use the \texttt{np.random.standard\_normal()} method.

    \begin{tcolorbox}[breakable, size=fbox, boxrule=1pt, pad at break*=1mm,colback=cellbackground, colframe=cellborder]
\prompt{In}{incolor}{3}{\boxspacing}
\begin{Verbatim}[commandchars=\\\{\}]
\PY{c+c1}{\PYZsh{} define simulation function}
\PY{k}{def} \PY{n+nf}{simulate\PYZus{}path}\PY{p}{(}\PY{n}{s0}\PY{p}{,} \PY{n}{risk\PYZus{}free\PYZus{}rate}\PY{p}{,} \PY{n}{vol}\PY{p}{,} \PY{n}{horizon}\PY{p}{,} \PY{n}{timesteps}\PY{p}{,} \PY{n}{n\PYZus{}sims}\PY{p}{)}\PY{p}{:}
    \PY{c+c1}{\PYZsh{} choose the random seed}
    \PY{n}{seed} \PY{o}{=} \PY{l+m+mi}{2023}
    \PY{n}{rng} \PY{o}{=} \PY{n}{np}\PY{o}{.}\PY{n}{random}\PY{o}{.}\PY{n}{default\PYZus{}rng}\PY{p}{(}\PY{n}{seed}\PY{p}{)}
    \PY{c+c1}{\PYZsh{} read the params}
    \PY{n}{S0} \PY{o}{=} \PY{n}{s0} 
    \PY{n}{r} \PY{o}{=} \PY{n}{risk\PYZus{}free\PYZus{}rate}
    \PY{n}{T} \PY{o}{=} \PY{n}{horizon}
    \PY{n}{t} \PY{o}{=} \PY{n}{timesteps}
    \PY{n}{n} \PY{o}{=} \PY{n}{n\PYZus{}sims}
    \PY{c+c1}{\PYZsh{} define dt}
    \PY{n}{dt} \PY{o}{=} \PY{n}{T}\PY{o}{/}\PY{n}{t}
    \PY{c+c1}{\PYZsh{} simulate path}
    \PY{n}{S} \PY{o}{=} \PY{n}{np}\PY{o}{.}\PY{n}{zeros}\PY{p}{(}\PY{p}{(}\PY{n}{t}\PY{p}{,}\PY{n}{n}\PY{p}{)}\PY{p}{)}
    \PY{n}{S}\PY{p}{[}\PY{l+m+mi}{0}\PY{p}{]} \PY{o}{=} \PY{n}{S0}
    \PY{k}{for} \PY{n}{i} \PY{o+ow}{in} \PY{n+nb}{range}\PY{p}{(}\PY{l+m+mi}{0}\PY{p}{,} \PY{n}{t}\PY{o}{\PYZhy{}}\PY{l+m+mi}{1}\PY{p}{)}\PY{p}{:}
        \PY{n}{w} \PY{o}{=} \PY{n}{rng}\PY{o}{.}\PY{n}{standard\PYZus{}normal}\PY{p}{(}\PY{n}{n}\PY{p}{)}
        \PY{n}{S}\PY{p}{[}\PY{n}{i}\PY{o}{+}\PY{l+m+mi}{1}\PY{p}{]} \PY{o}{=} \PY{n}{S}\PY{p}{[}\PY{n}{i}\PY{p}{]} \PY{o}{*} \PY{p}{(}\PY{l+m+mi}{1}\PY{o}{+} \PY{n}{r}\PY{o}{*}\PY{n}{dt} \PY{o}{+} \PY{n}{vol}\PY{o}{*}\PY{n}{np}\PY{o}{.}\PY{n}{sqrt}\PY{p}{(}\PY{n}{dt}\PY{p}{)}\PY{o}{*}\PY{n}{w}\PY{p}{)}
    \PY{k}{return} \PY{n}{S}
\end{Verbatim}
\end{tcolorbox}

    We will choose:

\begin{enumerate}
\def\labelenumi{\arabic{enumi}.}
\item
  Number of paths: \texttt{100,000}
\item
  Time steps: \texttt{252} trading days
\end{enumerate}

    \begin{tcolorbox}[breakable, size=fbox, boxrule=1pt, pad at break*=1mm,colback=cellbackground, colframe=cellborder]
\prompt{In}{incolor}{4}{\boxspacing}
\begin{Verbatim}[commandchars=\\\{\}]
\PY{c+c1}{\PYZsh{} Monte Carlo parameters}
\PY{n}{n} \PY{o}{=} \PY{l+m+mi}{100000}
\PY{n}{t} \PY{o}{=} \PY{l+m+mi}{252}
\end{Verbatim}
\end{tcolorbox}

    \begin{tcolorbox}[breakable, size=fbox, boxrule=1pt, pad at break*=1mm,colback=cellbackground, colframe=cellborder]
\prompt{In}{incolor}{5}{\boxspacing}
\begin{Verbatim}[commandchars=\\\{\}]
\PY{c+c1}{\PYZsh{} Assign simulated price path to dataframe for analysis and plotting}
\PY{n}{S} \PY{o}{=} \PY{n}{pd}\PY{o}{.}\PY{n}{DataFrame}\PY{p}{(}\PY{n}{simulate\PYZus{}path}\PY{p}{(}\PY{n}{S0}\PY{p}{,}\PY{n}{risk\PYZus{}free\PYZus{}rate}\PY{p}{,}\PY{n}{vol}\PY{p}{,}\PY{n}{T}\PY{p}{,}\PY{n}{t}\PY{p}{,}\PY{n}{n}\PY{p}{)}\PY{p}{)}
\end{Verbatim}
\end{tcolorbox}

    \begin{tcolorbox}[breakable, size=fbox, boxrule=1pt, pad at break*=1mm,colback=cellbackground, colframe=cellborder]
\prompt{In}{incolor}{6}{\boxspacing}
\begin{Verbatim}[commandchars=\\\{\}]
\PY{c+c1}{\PYZsh{} Plot initial 100 simulated path using matplotlib}
\PY{n}{plt}\PY{o}{.}\PY{n}{plot}\PY{p}{(}\PY{n}{S}\PY{o}{.}\PY{n}{iloc}\PY{p}{[}\PY{p}{:}\PY{p}{,}\PY{p}{:}\PY{l+m+mi}{100}\PY{p}{]}\PY{p}{)}
\PY{n}{plt}\PY{o}{.}\PY{n}{xlabel}\PY{p}{(}\PY{l+s+s1}{\PYZsq{}}\PY{l+s+s1}{time steps}\PY{l+s+s1}{\PYZsq{}}\PY{p}{)}
\PY{n}{plt}\PY{o}{.}\PY{n}{xlim}\PY{p}{(}\PY{l+m+mi}{0}\PY{p}{,}\PY{n}{t}\PY{p}{)}
\PY{n}{plt}\PY{o}{.}\PY{n}{ylabel}\PY{p}{(}\PY{l+s+s1}{\PYZsq{}}\PY{l+s+s1}{index levels}\PY{l+s+s1}{\PYZsq{}}\PY{p}{)}
\PY{n}{plt}\PY{o}{.}\PY{n}{title}\PY{p}{(}\PY{l+s+s1}{\PYZsq{}}\PY{l+s+s1}{Monte Carlo Simulated Asset Prices}\PY{l+s+s1}{\PYZsq{}}\PY{p}{)}\PY{p}{;}
\end{Verbatim}
\end{tcolorbox}

    \begin{center}
    \adjustimage{max size={0.9\linewidth}{0.9\paperheight}}{exam3_files/exam3_21_0.png}
    \end{center}
    { \hspace*{\fill} \\}
    
    \subsection{Exotic Options Pricing}\label{exotic-options-pricing}

    Using the simulation from the previous section, I will calculate the
payoff of the options for all realizations of the asset path. Then, I
will take the expectation of the option price, discounted back to the
present value.

    \subsubsection{Asian Options Pricing}\label{asian-options-pricing}

    An Asian option is an option where the payoff depends on the average
price of the underlying asset over a certain period of time. Averaging
can be either be Arithmetic or Geometric. There are two types of Payoff
types: average rate, where averaging price is used in place of
underlying price; and erage strike, where averaging price is used in
place of strike \hyperref[3]{[3]}

Average strike call: \[
\max(S-A,0)
\]

Average strike put: \[
\max(A-S,0)
\]

Average rate call: \[
\max(A-E,0)
\]

Average rate put: \[
\max(E-A,0)
\]

    The Average tracking variable:

\[
A_{i}= \frac{1}{i} \sum_{k=1}^{i} S\left(t_{k}\right)
\]

Where \(i\) is the total number of sampling dates.

    In this function, I will use:

\begin{enumerate}
\def\labelenumi{\arabic{enumi}.}
\item
  Payoff Type: Average rate
\item
  Type of Average: Athrimetic Average
\end{enumerate}

    Under the risk-neutral framework, we assume the asset is going to earn,
on average, the risk-free interest rate. Hence, the option value at time
t would simply be the discounted value of the expected payoff.

    \textbf{The payoff of the options is given by}

\[
C(S,t) = e^{-r(T-t)}(\mathbb{E}[\max(A-E),0])
\]

    \begin{tcolorbox}[breakable, size=fbox, boxrule=1pt, pad at break*=1mm,colback=cellbackground, colframe=cellborder]
\prompt{In}{incolor}{7}{\boxspacing}
\begin{Verbatim}[commandchars=\\\{\}]
\PY{c+c1}{\PYZsh{} Average price}
\PY{n}{A} \PY{o}{=} \PY{n}{S}\PY{o}{.}\PY{n}{mean}\PY{p}{(}\PY{n}{axis}\PY{o}{=}\PY{l+m+mi}{0}\PY{p}{)}
\PY{n}{C0} \PY{o}{=} \PY{n}{np}\PY{o}{.}\PY{n}{exp}\PY{p}{(}\PY{o}{\PYZhy{}}\PY{n}{risk\PYZus{}free\PYZus{}rate}\PY{o}{*}\PY{n}{T}\PY{p}{)} \PY{o}{*} \PY{n}{np}\PY{o}{.}\PY{n}{mean}\PY{p}{(}\PY{n}{np}\PY{o}{.}\PY{n}{maximum}\PY{p}{(}\PY{n}{A}\PY{o}{\PYZhy{}}\PY{n}{E}\PY{p}{,}\PY{l+m+mi}{0}\PY{p}{)}\PY{p}{)}
\PY{n}{P0} \PY{o}{=} \PY{n}{np}\PY{o}{.}\PY{n}{exp}\PY{p}{(}\PY{o}{\PYZhy{}}\PY{n}{risk\PYZus{}free\PYZus{}rate}\PY{o}{*}\PY{n}{T}\PY{p}{)} \PY{o}{*} \PY{n}{np}\PY{o}{.}\PY{n}{mean}\PY{p}{(}\PY{n}{np}\PY{o}{.}\PY{n}{maximum}\PY{p}{(}\PY{n}{E}\PY{o}{\PYZhy{}}\PY{n}{A}\PY{p}{,}\PY{l+m+mi}{0}\PY{p}{)}\PY{p}{)}
\PY{c+c1}{\PYZsh{} Print the values}
\PY{n+nb}{print}\PY{p}{(}\PY{l+s+sa}{f}\PY{l+s+s2}{\PYZdq{}}\PY{l+s+s2}{Asian Call Option Value is }\PY{l+s+si}{\PYZob{}}\PY{n}{C0}\PY{l+s+si}{:}\PY{l+s+s2}{0.4f}\PY{l+s+si}{\PYZcb{}}\PY{l+s+s2}{\PYZdq{}}\PY{p}{)}
\PY{n+nb}{print}\PY{p}{(}\PY{l+s+sa}{f}\PY{l+s+s2}{\PYZdq{}}\PY{l+s+s2}{Asian Put Option Value is }\PY{l+s+si}{\PYZob{}}\PY{n}{P0}\PY{l+s+si}{:}\PY{l+s+s2}{0.4f}\PY{l+s+si}{\PYZcb{}}\PY{l+s+s2}{\PYZdq{}}\PY{p}{)}
\end{Verbatim}
\end{tcolorbox}

    \begin{Verbatim}[commandchars=\\\{\}]
Asian Call Option Value is 5.7603
Asian Put Option Value is 3.3464
    \end{Verbatim}

    \subsubsection{Lookback Options Pricing}\label{lookback-options-pricing}

    The lookback option has a payoff that depends on the maximum or minimum
of the realized asset price. There are two types of payoff: The rate and
the strike option, also called the \textbf{fixed strike} and the
\textbf{floating strike} respectively \hyperref[4]{[4]}.

    Variable \(M_{\max}\) as the realized maximum and \(M_{\min}\) as the
realized minimum of the asset from the start of the sampling period
\(t = 0\) until the current time \(t\):

\[
\begin{align*}
M_{\max} &= \max_{0 \leq \tau \leq t} S(\tau)
\\\\
M_{\min} &= \min_{0 \leq \tau \leq t} S(\tau)
\end{align*}
\]

    The payoff function becomes

Floating strike lookback call: \[
\max(S - M_{\min},0)
\]

Floating strike lookback put: \[
\max( M_{\max} - S,0)
\]

Fixed strike lookback call: \[
\max(M_{\max} - E,0)
\]

Fixed strike lookback put: \[
\max(E - M_{\min},0)
\]

    The value of our lookback option is a function of three variables,
\(V (S, M, t)\). In this function, I will use:

\begin{enumerate}
\def\labelenumi{\arabic{enumi}.}
\item
  Payoff Type: Fixed strike
\item
  Maximum measurement: Continous
\end{enumerate}

The algorithm is to price option of each path and then calculate the
average payoff over all realizations. Take the present value of this
average which is the value of the option.

    \begin{tcolorbox}[breakable, size=fbox, boxrule=1pt, pad at break*=1mm,colback=cellbackground, colframe=cellborder]
\prompt{In}{incolor}{8}{\boxspacing}
\begin{Verbatim}[commandchars=\\\{\}]
\PY{c+c1}{\PYZsh{} Tracking Variable for Maximum and Minimum}
\PY{n}{M\PYZus{}max} \PY{o}{=} \PY{n}{np}\PY{o}{.}\PY{n}{max}\PY{p}{(}\PY{n}{S}\PY{p}{,}\PY{n}{axis}\PY{o}{=}\PY{l+m+mi}{0}\PY{p}{)}
\PY{n}{M\PYZus{}min} \PY{o}{=} \PY{n}{np}\PY{o}{.}\PY{n}{min}\PY{p}{(}\PY{n}{S}\PY{p}{,}\PY{n}{axis}\PY{o}{=}\PY{l+m+mi}{0}\PY{p}{)}
\PY{c+c1}{\PYZsh{} Pricing Options}
\PY{n}{C0} \PY{o}{=} \PY{n}{np}\PY{o}{.}\PY{n}{exp}\PY{p}{(}\PY{o}{\PYZhy{}}\PY{n}{risk\PYZus{}free\PYZus{}rate}\PY{o}{*}\PY{n}{T}\PY{p}{)} \PY{o}{*} \PY{n}{np}\PY{o}{.}\PY{n}{mean}\PY{p}{(}\PY{n}{np}\PY{o}{.}\PY{n}{maximum}\PY{p}{(}\PY{n}{M\PYZus{}max}\PY{o}{\PYZhy{}}\PY{n}{E}\PY{p}{,}\PY{l+m+mi}{0}\PY{p}{)}\PY{p}{)}
\PY{n}{P0} \PY{o}{=} \PY{n}{np}\PY{o}{.}\PY{n}{exp}\PY{p}{(}\PY{o}{\PYZhy{}}\PY{n}{risk\PYZus{}free\PYZus{}rate}\PY{o}{*}\PY{n}{T}\PY{p}{)} \PY{o}{*} \PY{n}{np}\PY{o}{.}\PY{n}{mean}\PY{p}{(}\PY{n}{np}\PY{o}{.}\PY{n}{maximum}\PY{p}{(}\PY{n}{E}\PY{o}{\PYZhy{}}\PY{n}{M\PYZus{}min}\PY{p}{,}\PY{l+m+mi}{0}\PY{p}{)}\PY{p}{)}
\PY{c+c1}{\PYZsh{} Print the values}
\PY{n+nb}{print}\PY{p}{(}\PY{l+s+sa}{f}\PY{l+s+s2}{\PYZdq{}}\PY{l+s+s2}{Lookback Call Option Value is }\PY{l+s+si}{\PYZob{}}\PY{n}{C0}\PY{l+s+si}{:}\PY{l+s+s2}{0.4f}\PY{l+s+si}{\PYZcb{}}\PY{l+s+s2}{\PYZdq{}}\PY{p}{)}
\PY{n+nb}{print}\PY{p}{(}\PY{l+s+sa}{f}\PY{l+s+s2}{\PYZdq{}}\PY{l+s+s2}{Lookback Put Option Value is }\PY{l+s+si}{\PYZob{}}\PY{n}{P0}\PY{l+s+si}{:}\PY{l+s+s2}{0.4f}\PY{l+s+si}{\PYZcb{}}\PY{l+s+s2}{\PYZdq{}}\PY{p}{)}
\end{Verbatim}
\end{tcolorbox}

    \begin{Verbatim}[commandchars=\\\{\}]
Lookback Call Option Value is 18.3002
Lookback Put Option Value is 11.7159
    \end{Verbatim}

    \section{Varying the Initial Data}\label{varying-the-initial-data}

    Using previous code for our Monte Carlo simulation and Asian options
pricing, I will create an \texttt{asian} function to price an option
values for different parameters.

    \begin{tcolorbox}[breakable, size=fbox, boxrule=1pt, pad at break*=1mm,colback=cellbackground, colframe=cellborder]
\prompt{In}{incolor}{9}{\boxspacing}
\begin{Verbatim}[commandchars=\\\{\}]
\PY{c+c1}{\PYZsh{} Setting the random seed}
\PY{n}{seed} \PY{o}{=} \PY{l+m+mi}{2023}
\PY{n}{rng} \PY{o}{=} \PY{n}{np}\PY{o}{.}\PY{n}{random}\PY{o}{.}\PY{n}{default\PYZus{}rng}\PY{p}{(}\PY{n}{seed}\PY{p}{)}
\end{Verbatim}
\end{tcolorbox}

    \begin{tcolorbox}[breakable, size=fbox, boxrule=1pt, pad at break*=1mm,colback=cellbackground, colframe=cellborder]
\prompt{In}{incolor}{10}{\boxspacing}
\begin{Verbatim}[commandchars=\\\{\}]
\PY{k}{def} \PY{n+nf}{asian}\PY{p}{(}\PY{n}{s0}\PY{p}{,} \PY{n}{strike}\PY{p}{,} \PY{n}{risk\PYZus{}free\PYZus{}rate}\PY{p}{,} \PY{n}{vol}\PY{p}{,} \PY{n}{horizon}\PY{p}{,} \PY{n}{timesteps}\PY{p}{,} \PY{n}{n\PYZus{}sims}\PY{p}{,} \PY{n}{option\PYZus{}type} \PY{o}{=} \PY{l+s+s2}{\PYZdq{}}\PY{l+s+s2}{C}\PY{l+s+s2}{\PYZdq{}}\PY{p}{)}\PY{p}{:}
    \PY{c+c1}{\PYZsh{} Euler \PYZhy{} Maruyama Scheme}
    \PY{n}{S0} \PY{o}{=} \PY{n}{s0}
    \PY{n}{E} \PY{o}{=} \PY{n}{strike} 
    \PY{n}{r} \PY{o}{=} \PY{n}{risk\PYZus{}free\PYZus{}rate}
    \PY{n}{T} \PY{o}{=} \PY{n}{horizon}
    \PY{n}{t} \PY{o}{=} \PY{n}{timesteps}
    \PY{n}{n} \PY{o}{=} \PY{n}{n\PYZus{}sims}
    \PY{n}{dt} \PY{o}{=} \PY{n}{T}\PY{o}{/}\PY{n}{t}
    \PY{n}{S} \PY{o}{=} \PY{n}{np}\PY{o}{.}\PY{n}{zeros}\PY{p}{(}\PY{p}{(}\PY{n}{t}\PY{p}{,}\PY{n}{n}\PY{p}{)}\PY{p}{)}
    \PY{n}{S}\PY{p}{[}\PY{l+m+mi}{0}\PY{p}{]} \PY{o}{=} \PY{n}{S0}
    \PY{k}{for} \PY{n}{i} \PY{o+ow}{in} \PY{n+nb}{range}\PY{p}{(}\PY{l+m+mi}{0}\PY{p}{,} \PY{n}{t}\PY{o}{\PYZhy{}}\PY{l+m+mi}{1}\PY{p}{)}\PY{p}{:}
        \PY{n}{w} \PY{o}{=} \PY{n}{rng}\PY{o}{.}\PY{n}{standard\PYZus{}normal}\PY{p}{(}\PY{n}{n}\PY{p}{)}
        \PY{n}{S}\PY{p}{[}\PY{n}{i}\PY{o}{+}\PY{l+m+mi}{1}\PY{p}{]} \PY{o}{=} \PY{n}{S}\PY{p}{[}\PY{n}{i}\PY{p}{]} \PY{o}{*} \PY{p}{(}\PY{l+m+mi}{1}\PY{o}{+} \PY{n}{r}\PY{o}{*}\PY{n}{dt} \PY{o}{+} \PY{n}{vol}\PY{o}{*}\PY{n}{np}\PY{o}{.}\PY{n}{sqrt}\PY{p}{(}\PY{n}{dt}\PY{p}{)}\PY{o}{*}\PY{n}{w}\PY{p}{)}
    \PY{n}{S} \PY{o}{=} \PY{n}{pd}\PY{o}{.}\PY{n}{DataFrame}\PY{p}{(}\PY{n}{simulate\PYZus{}path}\PY{p}{(}\PY{n}{S0}\PY{p}{,}\PY{n}{risk\PYZus{}free\PYZus{}rate}\PY{p}{,}\PY{n}{vol}\PY{p}{,}\PY{n}{T}\PY{p}{,}\PY{n}{t}\PY{p}{,}\PY{n}{n}\PY{p}{)}\PY{p}{)}    \PY{c+c1}{\PYZsh{} Building a Dataframe for all simulated path}
    \PY{c+c1}{\PYZsh{} Average price}
    \PY{n}{A} \PY{o}{=} \PY{n}{S}\PY{o}{.}\PY{n}{mean}\PY{p}{(}\PY{n}{axis}\PY{o}{=}\PY{l+m+mi}{0}\PY{p}{)}
    \PY{c+c1}{\PYZsh{} Option pricing Algorithm}
    \PY{k}{if} \PY{n}{option\PYZus{}type} \PY{o}{==} \PY{l+s+s2}{\PYZdq{}}\PY{l+s+s2}{C}\PY{l+s+s2}{\PYZdq{}}\PY{p}{:}
        \PY{k}{return} \PY{n}{np}\PY{o}{.}\PY{n}{exp}\PY{p}{(}\PY{o}{\PYZhy{}}\PY{n}{r}\PY{o}{*}\PY{n}{T}\PY{p}{)} \PY{o}{*} \PY{n}{np}\PY{o}{.}\PY{n}{mean}\PY{p}{(}\PY{n}{np}\PY{o}{.}\PY{n}{maximum}\PY{p}{(}\PY{n}{A}\PY{o}{\PYZhy{}}\PY{n}{E}\PY{p}{,}\PY{l+m+mi}{0}\PY{p}{)}\PY{p}{)}
    \PY{k}{elif} \PY{n}{option\PYZus{}type} \PY{o}{==} \PY{l+s+s2}{\PYZdq{}}\PY{l+s+s2}{P}\PY{l+s+s2}{\PYZdq{}}\PY{p}{:}
        \PY{k}{return} \PY{n}{np}\PY{o}{.}\PY{n}{exp}\PY{p}{(}\PY{o}{\PYZhy{}}\PY{n}{r}\PY{o}{*}\PY{n}{T}\PY{p}{)} \PY{o}{*} \PY{n}{np}\PY{o}{.}\PY{n}{mean}\PY{p}{(}\PY{n}{np}\PY{o}{.}\PY{n}{maximum}\PY{p}{(}\PY{n}{E}\PY{o}{\PYZhy{}}\PY{n}{A}\PY{p}{,}\PY{l+m+mi}{0}\PY{p}{)}\PY{p}{)}
    \PY{k}{else}\PY{p}{:}
        \PY{n+nb}{print}\PY{p}{(}\PY{l+s+s2}{\PYZdq{}}\PY{l+s+s2}{Wrong Option Type. Use `C` for Call option and `P` for Put option.}\PY{l+s+s2}{\PYZdq{}}\PY{p}{)}
\end{Verbatim}
\end{tcolorbox}

    And doing the same to create \texttt{lookback} function to price
Lookback options

    \begin{tcolorbox}[breakable, size=fbox, boxrule=1pt, pad at break*=1mm,colback=cellbackground, colframe=cellborder]
\prompt{In}{incolor}{11}{\boxspacing}
\begin{Verbatim}[commandchars=\\\{\}]
\PY{k}{def} \PY{n+nf}{lookback}\PY{p}{(}\PY{n}{s0}\PY{p}{,} \PY{n}{strike}\PY{p}{,} \PY{n}{risk\PYZus{}free\PYZus{}rate}\PY{p}{,} \PY{n}{vol}\PY{p}{,} \PY{n}{horizon}\PY{p}{,} \PY{n}{timesteps}\PY{p}{,} \PY{n}{n\PYZus{}sims}\PY{p}{,} \PY{n}{option\PYZus{}type} \PY{o}{=} \PY{l+s+s2}{\PYZdq{}}\PY{l+s+s2}{C}\PY{l+s+s2}{\PYZdq{}}\PY{p}{)}\PY{p}{:}
    \PY{c+c1}{\PYZsh{} Euler \PYZhy{} Maruyama Scheme}
    \PY{n}{S0} \PY{o}{=} \PY{n}{s0}
    \PY{n}{E} \PY{o}{=} \PY{n}{strike}
    \PY{n}{r} \PY{o}{=} \PY{n}{risk\PYZus{}free\PYZus{}rate}
    \PY{n}{T} \PY{o}{=} \PY{n}{horizon}
    \PY{n}{t} \PY{o}{=} \PY{n}{timesteps}
    \PY{n}{n} \PY{o}{=} \PY{n}{n\PYZus{}sims}
    \PY{n}{dt} \PY{o}{=} \PY{n}{T}\PY{o}{/}\PY{n}{t}
    \PY{n}{S} \PY{o}{=} \PY{n}{np}\PY{o}{.}\PY{n}{zeros}\PY{p}{(}\PY{p}{(}\PY{n}{t}\PY{p}{,}\PY{n}{n}\PY{p}{)}\PY{p}{)}
    \PY{n}{S}\PY{p}{[}\PY{l+m+mi}{0}\PY{p}{]} \PY{o}{=} \PY{n}{S0}
    \PY{k}{for} \PY{n}{i} \PY{o+ow}{in} \PY{n+nb}{range}\PY{p}{(}\PY{l+m+mi}{0}\PY{p}{,} \PY{n}{t}\PY{o}{\PYZhy{}}\PY{l+m+mi}{1}\PY{p}{)}\PY{p}{:}
        \PY{n}{w} \PY{o}{=} \PY{n}{rng}\PY{o}{.}\PY{n}{standard\PYZus{}normal}\PY{p}{(}\PY{n}{n}\PY{p}{)}
        \PY{n}{S}\PY{p}{[}\PY{n}{i}\PY{o}{+}\PY{l+m+mi}{1}\PY{p}{]} \PY{o}{=} \PY{n}{S}\PY{p}{[}\PY{n}{i}\PY{p}{]} \PY{o}{*} \PY{p}{(}\PY{l+m+mi}{1}\PY{o}{+} \PY{n}{r}\PY{o}{*}\PY{n}{dt} \PY{o}{+} \PY{n}{vol}\PY{o}{*}\PY{n}{np}\PY{o}{.}\PY{n}{sqrt}\PY{p}{(}\PY{n}{dt}\PY{p}{)}\PY{o}{*}\PY{n}{w}\PY{p}{)}
    \PY{n}{S} \PY{o}{=} \PY{n}{pd}\PY{o}{.}\PY{n}{DataFrame}\PY{p}{(}\PY{n}{simulate\PYZus{}path}\PY{p}{(}\PY{n}{S0}\PY{p}{,}\PY{n}{risk\PYZus{}free\PYZus{}rate}\PY{p}{,}\PY{n}{vol}\PY{p}{,}\PY{n}{T}\PY{p}{,}\PY{n}{t}\PY{p}{,}\PY{n}{n}\PY{p}{)}\PY{p}{)}    \PY{c+c1}{\PYZsh{} Building a Dataframe for all simulated path}
    
    \PY{c+c1}{\PYZsh{} Tracking Variable for Maximum and Minimum}
    \PY{n}{M\PYZus{}max} \PY{o}{=} \PY{n}{np}\PY{o}{.}\PY{n}{max}\PY{p}{(}\PY{n}{S}\PY{p}{,}\PY{n}{axis}\PY{o}{=}\PY{l+m+mi}{0}\PY{p}{)}
    \PY{n}{M\PYZus{}min} \PY{o}{=} \PY{n}{np}\PY{o}{.}\PY{n}{min}\PY{p}{(}\PY{n}{S}\PY{p}{,}\PY{n}{axis}\PY{o}{=}\PY{l+m+mi}{0}\PY{p}{)}
    
    \PY{c+c1}{\PYZsh{} Discounted back to PV}
    \PY{k}{if} \PY{n}{option\PYZus{}type} \PY{o}{==} \PY{l+s+s2}{\PYZdq{}}\PY{l+s+s2}{C}\PY{l+s+s2}{\PYZdq{}}\PY{p}{:}
        \PY{k}{return} \PY{n}{np}\PY{o}{.}\PY{n}{exp}\PY{p}{(}\PY{o}{\PYZhy{}}\PY{n}{r}\PY{o}{*}\PY{n}{T}\PY{p}{)} \PY{o}{*} \PY{n}{np}\PY{o}{.}\PY{n}{mean}\PY{p}{(}\PY{n}{np}\PY{o}{.}\PY{n}{maximum}\PY{p}{(}\PY{n}{M\PYZus{}max}\PY{o}{\PYZhy{}}\PY{n}{E}\PY{p}{,}\PY{l+m+mi}{0}\PY{p}{)}\PY{p}{)}
    \PY{k}{elif} \PY{n}{option\PYZus{}type} \PY{o}{==} \PY{l+s+s2}{\PYZdq{}}\PY{l+s+s2}{P}\PY{l+s+s2}{\PYZdq{}}\PY{p}{:}
        \PY{k}{return} \PY{n}{np}\PY{o}{.}\PY{n}{exp}\PY{p}{(}\PY{o}{\PYZhy{}}\PY{n}{risk\PYZus{}free\PYZus{}rate}\PY{o}{*}\PY{n}{T}\PY{p}{)} \PY{o}{*} \PY{n}{np}\PY{o}{.}\PY{n}{mean}\PY{p}{(}\PY{n}{np}\PY{o}{.}\PY{n}{maximum}\PY{p}{(}\PY{n}{E}\PY{o}{\PYZhy{}}\PY{n}{M\PYZus{}min}\PY{p}{,}\PY{l+m+mi}{0}\PY{p}{)}\PY{p}{)}
    \PY{k}{else}\PY{p}{:}
        \PY{n+nb}{print}\PY{p}{(}\PY{l+s+s2}{\PYZdq{}}\PY{l+s+s2}{Wrong Option Type. Use `C` for Call option and `P` for Put option.}\PY{l+s+s2}{\PYZdq{}}\PY{p}{)}
\end{Verbatim}
\end{tcolorbox}

    Testing the function with our previous results for Asian Options

    \begin{tcolorbox}[breakable, size=fbox, boxrule=1pt, pad at break*=1mm,colback=cellbackground, colframe=cellborder]
\prompt{In}{incolor}{12}{\boxspacing}
\begin{Verbatim}[commandchars=\\\{\}]
\PY{n}{asian}\PY{p}{(}\PY{l+m+mi}{100}\PY{p}{,}\PY{l+m+mi}{100}\PY{p}{,}\PY{l+m+mf}{0.05}\PY{p}{,}\PY{l+m+mf}{0.2}\PY{p}{,}\PY{l+m+mi}{1}\PY{p}{,}\PY{l+m+mi}{252}\PY{p}{,}\PY{l+m+mi}{100000}\PY{p}{,}\PY{l+s+s2}{\PYZdq{}}\PY{l+s+s2}{C}\PY{l+s+s2}{\PYZdq{}}\PY{p}{)}\PY{p}{,} \PY{n}{asian}\PY{p}{(}\PY{l+m+mi}{100}\PY{p}{,}\PY{l+m+mi}{100}\PY{p}{,}\PY{l+m+mf}{0.05}\PY{p}{,}\PY{l+m+mf}{0.2}\PY{p}{,}\PY{l+m+mi}{1}\PY{p}{,}\PY{l+m+mi}{252}\PY{p}{,}\PY{l+m+mi}{100000}\PY{p}{,}\PY{l+s+s2}{\PYZdq{}}\PY{l+s+s2}{P}\PY{l+s+s2}{\PYZdq{}}\PY{p}{)}
\end{Verbatim}
\end{tcolorbox}

            \begin{tcolorbox}[breakable, size=fbox, boxrule=.5pt, pad at break*=1mm, opacityfill=0]
\prompt{Out}{outcolor}{12}{\boxspacing}
\begin{Verbatim}[commandchars=\\\{\}]
(5.760317815971038, 3.346444480615885)
\end{Verbatim}
\end{tcolorbox}
        
    And for the Lookback Options

    \begin{tcolorbox}[breakable, size=fbox, boxrule=1pt, pad at break*=1mm,colback=cellbackground, colframe=cellborder]
\prompt{In}{incolor}{13}{\boxspacing}
\begin{Verbatim}[commandchars=\\\{\}]
\PY{n}{lookback}\PY{p}{(}\PY{l+m+mi}{100}\PY{p}{,}\PY{l+m+mi}{100}\PY{p}{,}\PY{l+m+mf}{0.05}\PY{p}{,}\PY{l+m+mf}{0.2}\PY{p}{,}\PY{l+m+mi}{1}\PY{p}{,}\PY{l+m+mi}{252}\PY{p}{,}\PY{l+m+mi}{100000}\PY{p}{,}\PY{l+s+s2}{\PYZdq{}}\PY{l+s+s2}{C}\PY{l+s+s2}{\PYZdq{}}\PY{p}{)}\PY{p}{,} \PY{n}{lookback}\PY{p}{(}\PY{l+m+mi}{100}\PY{p}{,}\PY{l+m+mi}{100}\PY{p}{,}\PY{l+m+mf}{0.05}\PY{p}{,}\PY{l+m+mf}{0.2}\PY{p}{,}\PY{l+m+mi}{1}\PY{p}{,}\PY{l+m+mi}{252}\PY{p}{,}\PY{l+m+mi}{100000}\PY{p}{,}\PY{l+s+s2}{\PYZdq{}}\PY{l+s+s2}{P}\PY{l+s+s2}{\PYZdq{}}\PY{p}{)}
\end{Verbatim}
\end{tcolorbox}

            \begin{tcolorbox}[breakable, size=fbox, boxrule=.5pt, pad at break*=1mm, opacityfill=0]
\prompt{Out}{outcolor}{13}{\boxspacing}
\begin{Verbatim}[commandchars=\\\{\}]
(18.300180586891152, 11.715854637463458)
\end{Verbatim}
\end{tcolorbox}
        
    Both functions are working as expected. In the next section, I will vary
the parameters of the functions and observe the effect on option prices.

    \subsection{\texorpdfstring{Vary the Volatility
\(\sigma\)}{Vary the Volatility \textbackslash sigma}}\label{vary-the-volatility-sigma}

    Now, let's vary the volatility for three different levels:
\(\sigma_1 = 10%
\), \(\sigma_2 = 20%
\), and \(\sigma_3 = 30%
\).

    \begin{tcolorbox}[breakable, size=fbox, boxrule=1pt, pad at break*=1mm,colback=cellbackground, colframe=cellborder]
\prompt{In}{incolor}{14}{\boxspacing}
\begin{Verbatim}[commandchars=\\\{\}]
\PY{n}{vol\PYZus{}set} \PY{o}{=} \PY{p}{[}\PY{l+m+mf}{0.15}\PY{p}{,}\PY{l+m+mf}{0.2}\PY{p}{,}\PY{l+m+mf}{0.25}\PY{p}{]}
\PY{n}{asian\PYZus{}calls} \PY{o}{=} \PY{p}{[}\PY{n}{asian}\PY{p}{(}\PY{l+m+mi}{100}\PY{p}{,}\PY{l+m+mi}{100}\PY{p}{,}\PY{l+m+mf}{0.05}\PY{p}{,}\PY{n}{vol}\PY{p}{,}\PY{l+m+mi}{1}\PY{p}{,}\PY{l+m+mi}{252}\PY{p}{,}\PY{l+m+mi}{100000}\PY{p}{,}\PY{l+s+s2}{\PYZdq{}}\PY{l+s+s2}{C}\PY{l+s+s2}{\PYZdq{}}\PY{p}{)} \PY{k}{for} \PY{n}{vol} \PY{o+ow}{in} \PY{n}{vol\PYZus{}set}\PY{p}{]}
\PY{n}{asian\PYZus{}puts} \PY{o}{=} \PY{p}{[}\PY{n}{asian}\PY{p}{(}\PY{l+m+mi}{100}\PY{p}{,}\PY{l+m+mi}{100}\PY{p}{,}\PY{l+m+mf}{0.05}\PY{p}{,}\PY{n}{vol}\PY{p}{,}\PY{l+m+mi}{1}\PY{p}{,}\PY{l+m+mi}{252}\PY{p}{,}\PY{l+m+mi}{100000}\PY{p}{,}\PY{l+s+s2}{\PYZdq{}}\PY{l+s+s2}{P}\PY{l+s+s2}{\PYZdq{}}\PY{p}{)} \PY{k}{for} \PY{n}{vol} \PY{o+ow}{in} \PY{n}{vol\PYZus{}set}\PY{p}{]}
\PY{n}{lookback\PYZus{}calls} \PY{o}{=} \PY{p}{[}\PY{n}{lookback}\PY{p}{(}\PY{l+m+mi}{100}\PY{p}{,}\PY{l+m+mi}{100}\PY{p}{,}\PY{l+m+mf}{0.05}\PY{p}{,}\PY{n}{vol}\PY{p}{,}\PY{l+m+mi}{1}\PY{p}{,}\PY{l+m+mi}{252}\PY{p}{,}\PY{l+m+mi}{100000}\PY{p}{,}\PY{l+s+s2}{\PYZdq{}}\PY{l+s+s2}{C}\PY{l+s+s2}{\PYZdq{}}\PY{p}{)} \PY{k}{for} \PY{n}{vol} \PY{o+ow}{in} \PY{n}{vol\PYZus{}set}\PY{p}{]}
\PY{n}{lookback\PYZus{}puts} \PY{o}{=} \PY{p}{[}\PY{n}{lookback}\PY{p}{(}\PY{l+m+mi}{100}\PY{p}{,}\PY{l+m+mi}{100}\PY{p}{,}\PY{l+m+mf}{0.05}\PY{p}{,}\PY{n}{vol}\PY{p}{,}\PY{l+m+mi}{1}\PY{p}{,}\PY{l+m+mi}{252}\PY{p}{,}\PY{l+m+mi}{100000}\PY{p}{,}\PY{l+s+s2}{\PYZdq{}}\PY{l+s+s2}{P}\PY{l+s+s2}{\PYZdq{}}\PY{p}{)} \PY{k}{for} \PY{n}{vol} \PY{o+ow}{in} \PY{n}{vol\PYZus{}set}\PY{p}{]}
\end{Verbatim}
\end{tcolorbox}

    \begin{tcolorbox}[breakable, size=fbox, boxrule=1pt, pad at break*=1mm,colback=cellbackground, colframe=cellborder]
\prompt{In}{incolor}{15}{\boxspacing}
\begin{Verbatim}[commandchars=\\\{\}]
\PY{c+c1}{\PYZsh{} Create a dictionary with the sets and headers}
\PY{n}{data1} \PY{o}{=} \PY{p}{\PYZob{}}
    \PY{l+s+s1}{\PYZsq{}}\PY{l+s+s1}{Volatility}\PY{l+s+s1}{\PYZsq{}}\PY{p}{:} \PY{n}{vol\PYZus{}set}\PY{p}{,}
    \PY{l+s+s1}{\PYZsq{}}\PY{l+s+s1}{Asian Call}\PY{l+s+s1}{\PYZsq{}}\PY{p}{:} \PY{n}{asian\PYZus{}calls}\PY{p}{,}
    \PY{l+s+s1}{\PYZsq{}}\PY{l+s+s1}{Asian Put}\PY{l+s+s1}{\PYZsq{}}\PY{p}{:} \PY{n}{asian\PYZus{}puts}\PY{p}{,}
    \PY{l+s+s1}{\PYZsq{}}\PY{l+s+s1}{Lookback Call}\PY{l+s+s1}{\PYZsq{}}\PY{p}{:} \PY{n}{lookback\PYZus{}calls}\PY{p}{,}
    \PY{l+s+s1}{\PYZsq{}}\PY{l+s+s1}{Lookback Put}\PY{l+s+s1}{\PYZsq{}}\PY{p}{:} \PY{n}{lookback\PYZus{}puts}
\PY{p}{\PYZcb{}}
\PY{c+c1}{\PYZsh{} Create a DataFrame}
\PY{n}{df1} \PY{o}{=} \PY{n}{pd}\PY{o}{.}\PY{n}{DataFrame}\PY{p}{(}\PY{n}{data1}\PY{p}{)}\PY{o}{.}\PY{n}{set\PYZus{}index}\PY{p}{(}\PY{l+s+s1}{\PYZsq{}}\PY{l+s+s1}{Volatility}\PY{l+s+s1}{\PYZsq{}}\PY{p}{)}

\PY{n}{df1}
\end{Verbatim}
\end{tcolorbox}

            \begin{tcolorbox}[breakable, size=fbox, boxrule=.5pt, pad at break*=1mm, opacityfill=0]
\prompt{Out}{outcolor}{15}{\boxspacing}
\begin{Verbatim}[commandchars=\\\{\}]
            Asian Call  Asian Put  Lookback Call  Lookback Put
Volatility
0.15          4.683266   2.271560      14.255136      8.430421
0.20          5.760318   3.346444      18.300181     11.715855
0.25          6.849325   4.432830      22.482347     14.921957
\end{Verbatim}
\end{tcolorbox}
        
    It appears that both Asian and Lookback options prices increase when
volatility rises and decrease when volatility falls. However, it seems
that Asian options are less sensitive to changes in volatility. This
could be due to the effect of averaging the price of the underlying,
which mitigates the impact of sudden changes.

The price of Lookback options is increasing significantly. This could be
attributed to the high volatility that increases the probability of
extreme prices, consequently raising the maximum value \(M_{\max}\) and
minimum value \(M_{\min}\), thereby increasing the price of the options.

    \subsection{\texorpdfstring{Vary the inital stock price
\(S_0\)}{Vary the inital stock price S\_0}}\label{vary-the-inital-stock-price-s_0}

    Let's vary the initial stock price at three different levels: 90, 100,
and 110.

    \begin{tcolorbox}[breakable, size=fbox, boxrule=1pt, pad at break*=1mm,colback=cellbackground, colframe=cellborder]
\prompt{In}{incolor}{16}{\boxspacing}
\begin{Verbatim}[commandchars=\\\{\}]
\PY{n}{S0\PYZus{}set} \PY{o}{=} \PY{p}{[}\PY{l+m+mi}{90}\PY{p}{,}\PY{l+m+mi}{100}\PY{p}{,}\PY{l+m+mi}{110}\PY{p}{]}
\PY{n}{asian\PYZus{}calls} \PY{o}{=} \PY{p}{[}\PY{n}{asian}\PY{p}{(}\PY{n}{S0}\PY{p}{,}\PY{l+m+mi}{100}\PY{p}{,}\PY{l+m+mf}{0.05}\PY{p}{,}\PY{l+m+mf}{0.2}\PY{p}{,}\PY{l+m+mi}{1}\PY{p}{,}\PY{l+m+mi}{252}\PY{p}{,}\PY{l+m+mi}{100000}\PY{p}{,}\PY{l+s+s2}{\PYZdq{}}\PY{l+s+s2}{C}\PY{l+s+s2}{\PYZdq{}}\PY{p}{)} \PY{k}{for} \PY{n}{S0} \PY{o+ow}{in} \PY{n}{S0\PYZus{}set}\PY{p}{]}
\PY{n}{asian\PYZus{}puts} \PY{o}{=} \PY{p}{[}\PY{n}{asian}\PY{p}{(}\PY{n}{S0}\PY{p}{,}\PY{l+m+mi}{100}\PY{p}{,}\PY{l+m+mf}{0.05}\PY{p}{,}\PY{l+m+mf}{0.2}\PY{p}{,}\PY{l+m+mi}{1}\PY{p}{,}\PY{l+m+mi}{252}\PY{p}{,}\PY{l+m+mi}{100000}\PY{p}{,}\PY{l+s+s2}{\PYZdq{}}\PY{l+s+s2}{P}\PY{l+s+s2}{\PYZdq{}}\PY{p}{)} \PY{k}{for} \PY{n}{S0} \PY{o+ow}{in} \PY{n}{S0\PYZus{}set}\PY{p}{]}
\PY{n}{lookback\PYZus{}calls} \PY{o}{=} \PY{p}{[}\PY{n}{lookback}\PY{p}{(}\PY{n}{S0}\PY{p}{,}\PY{l+m+mi}{100}\PY{p}{,}\PY{l+m+mf}{0.05}\PY{p}{,}\PY{l+m+mf}{0.2}\PY{p}{,}\PY{l+m+mi}{1}\PY{p}{,}\PY{l+m+mi}{252}\PY{p}{,}\PY{l+m+mi}{100000}\PY{p}{,}\PY{l+s+s2}{\PYZdq{}}\PY{l+s+s2}{C}\PY{l+s+s2}{\PYZdq{}}\PY{p}{)} \PY{k}{for} \PY{n}{S0} \PY{o+ow}{in} \PY{n}{S0\PYZus{}set}\PY{p}{]}
\PY{n}{lookback\PYZus{}puts} \PY{o}{=} \PY{p}{[}\PY{n}{lookback}\PY{p}{(}\PY{n}{S0}\PY{p}{,}\PY{l+m+mi}{100}\PY{p}{,}\PY{l+m+mf}{0.05}\PY{p}{,}\PY{l+m+mf}{0.2}\PY{p}{,}\PY{l+m+mi}{1}\PY{p}{,}\PY{l+m+mi}{252}\PY{p}{,}\PY{l+m+mi}{100000}\PY{p}{,}\PY{l+s+s2}{\PYZdq{}}\PY{l+s+s2}{P}\PY{l+s+s2}{\PYZdq{}}\PY{p}{)} \PY{k}{for} \PY{n}{S0} \PY{o+ow}{in} \PY{n}{S0\PYZus{}set}\PY{p}{]}
\end{Verbatim}
\end{tcolorbox}

    \begin{tcolorbox}[breakable, size=fbox, boxrule=1pt, pad at break*=1mm,colback=cellbackground, colframe=cellborder]
\prompt{In}{incolor}{17}{\boxspacing}
\begin{Verbatim}[commandchars=\\\{\}]
\PY{c+c1}{\PYZsh{} Create a dictionary with the sets and headers}
\PY{n}{data2} \PY{o}{=} \PY{p}{\PYZob{}}
    \PY{l+s+s1}{\PYZsq{}}\PY{l+s+s1}{S0}\PY{l+s+s1}{\PYZsq{}}\PY{p}{:} \PY{n}{S0\PYZus{}set}\PY{p}{,}
    \PY{l+s+s1}{\PYZsq{}}\PY{l+s+s1}{Asian Call}\PY{l+s+s1}{\PYZsq{}}\PY{p}{:} \PY{n}{asian\PYZus{}calls}\PY{p}{,}
    \PY{l+s+s1}{\PYZsq{}}\PY{l+s+s1}{Asian Put}\PY{l+s+s1}{\PYZsq{}}\PY{p}{:} \PY{n}{asian\PYZus{}puts}\PY{p}{,}
    \PY{l+s+s1}{\PYZsq{}}\PY{l+s+s1}{Lookback Call}\PY{l+s+s1}{\PYZsq{}}\PY{p}{:} \PY{n}{lookback\PYZus{}calls}\PY{p}{,}
    \PY{l+s+s1}{\PYZsq{}}\PY{l+s+s1}{Lookback Put}\PY{l+s+s1}{\PYZsq{}}\PY{p}{:} \PY{n}{lookback\PYZus{}puts}
\PY{p}{\PYZcb{}}
\PY{c+c1}{\PYZsh{} Create a DataFrame}
\PY{n}{df2} \PY{o}{=} \PY{n}{pd}\PY{o}{.}\PY{n}{DataFrame}\PY{p}{(}\PY{n}{data2}\PY{p}{)}\PY{o}{.}\PY{n}{set\PYZus{}index}\PY{p}{(}\PY{l+s+s1}{\PYZsq{}}\PY{l+s+s1}{S0}\PY{l+s+s1}{\PYZsq{}}\PY{p}{)}

\PY{n}{df2}
\end{Verbatim}
\end{tcolorbox}

            \begin{tcolorbox}[breakable, size=fbox, boxrule=.5pt, pad at break*=1mm, opacityfill=0]
\prompt{Out}{outcolor}{17}{\boxspacing}
\begin{Verbatim}[commandchars=\\\{\}]
     Asian Call  Asian Put  Lookback Call  Lookback Put
S0
90     1.562839   8.902648       8.920297     20.056563
100    5.760318   3.346444      18.300181     11.715855
110   13.051017   0.883462      29.642493      5.699979
\end{Verbatim}
\end{tcolorbox}
        
    The price of Out of The Money (OTM) options at \(S_0 = 90\) is
significantly lower compared to At the Money (ATM) and In the Money
(ITM) options, as expected. For Asian Options, the price is notably
lower. This can be attributed to the effect of averaging the price of
the underlying, which makes it more challenging for the average to
increase.

In the case of Lookback Options, since the payoff depends on the maximum
and minimum of the underlying, it appears that the price of the options
doesn't change as drastically

    \subsection{\texorpdfstring{Vary the strike price
\(E\)}{Vary the strike price E}}\label{vary-the-strike-price-e}

    Let's vary the strike price at three different levels: 90, 100, and 110.

    \begin{tcolorbox}[breakable, size=fbox, boxrule=1pt, pad at break*=1mm,colback=cellbackground, colframe=cellborder]
\prompt{In}{incolor}{18}{\boxspacing}
\begin{Verbatim}[commandchars=\\\{\}]
\PY{n}{strike\PYZus{}set} \PY{o}{=} \PY{p}{[}\PY{l+m+mi}{90}\PY{p}{,}\PY{l+m+mi}{100}\PY{p}{,}\PY{l+m+mi}{110}\PY{p}{]}
\PY{n}{asian\PYZus{}calls} \PY{o}{=} \PY{p}{[}\PY{n}{asian}\PY{p}{(}\PY{l+m+mi}{100}\PY{p}{,}\PY{n}{strike}\PY{p}{,}\PY{l+m+mf}{0.05}\PY{p}{,}\PY{l+m+mf}{0.2}\PY{p}{,}\PY{l+m+mi}{1}\PY{p}{,}\PY{l+m+mi}{252}\PY{p}{,}\PY{l+m+mi}{100000}\PY{p}{,}\PY{l+s+s2}{\PYZdq{}}\PY{l+s+s2}{C}\PY{l+s+s2}{\PYZdq{}}\PY{p}{)} \PY{k}{for} \PY{n}{strike} \PY{o+ow}{in} \PY{n}{strike\PYZus{}set}\PY{p}{]}
\PY{n}{asian\PYZus{}puts} \PY{o}{=} \PY{p}{[}\PY{n}{asian}\PY{p}{(}\PY{l+m+mi}{100}\PY{p}{,}\PY{n}{strike}\PY{p}{,}\PY{l+m+mf}{0.05}\PY{p}{,}\PY{l+m+mf}{0.2}\PY{p}{,}\PY{l+m+mi}{1}\PY{p}{,}\PY{l+m+mi}{252}\PY{p}{,}\PY{l+m+mi}{100000}\PY{p}{,}\PY{l+s+s2}{\PYZdq{}}\PY{l+s+s2}{P}\PY{l+s+s2}{\PYZdq{}}\PY{p}{)} \PY{k}{for} \PY{n}{strike} \PY{o+ow}{in} \PY{n}{strike\PYZus{}set}\PY{p}{]}
\PY{n}{lookback\PYZus{}calls} \PY{o}{=} \PY{p}{[}\PY{n}{lookback}\PY{p}{(}\PY{l+m+mi}{100}\PY{p}{,}\PY{n}{strike}\PY{p}{,}\PY{l+m+mf}{0.05}\PY{p}{,}\PY{l+m+mf}{0.2}\PY{p}{,}\PY{l+m+mi}{1}\PY{p}{,}\PY{l+m+mi}{252}\PY{p}{,}\PY{l+m+mi}{100000}\PY{p}{,}\PY{l+s+s2}{\PYZdq{}}\PY{l+s+s2}{C}\PY{l+s+s2}{\PYZdq{}}\PY{p}{)} \PY{k}{for} \PY{n}{strike} \PY{o+ow}{in} \PY{n}{strike\PYZus{}set}\PY{p}{]}
\PY{n}{lookback\PYZus{}puts} \PY{o}{=} \PY{p}{[}\PY{n}{lookback}\PY{p}{(}\PY{l+m+mi}{100}\PY{p}{,}\PY{n}{strike}\PY{p}{,}\PY{l+m+mf}{0.05}\PY{p}{,}\PY{l+m+mf}{0.2}\PY{p}{,}\PY{l+m+mi}{1}\PY{p}{,}\PY{l+m+mi}{252}\PY{p}{,}\PY{l+m+mi}{100000}\PY{p}{,}\PY{l+s+s2}{\PYZdq{}}\PY{l+s+s2}{P}\PY{l+s+s2}{\PYZdq{}}\PY{p}{)} \PY{k}{for} \PY{n}{strike} \PY{o+ow}{in} \PY{n}{strike\PYZus{}set}\PY{p}{]}
\end{Verbatim}
\end{tcolorbox}

    \begin{tcolorbox}[breakable, size=fbox, boxrule=1pt, pad at break*=1mm,colback=cellbackground, colframe=cellborder]
\prompt{In}{incolor}{19}{\boxspacing}
\begin{Verbatim}[commandchars=\\\{\}]
\PY{c+c1}{\PYZsh{} Create a dictionary with the sets and headers}
\PY{n}{data3} \PY{o}{=} \PY{p}{\PYZob{}}
    \PY{l+s+s1}{\PYZsq{}}\PY{l+s+s1}{Strike}\PY{l+s+s1}{\PYZsq{}}\PY{p}{:} \PY{n}{strike\PYZus{}set}\PY{p}{,}
    \PY{l+s+s1}{\PYZsq{}}\PY{l+s+s1}{Asian Call}\PY{l+s+s1}{\PYZsq{}}\PY{p}{:} \PY{n}{asian\PYZus{}calls}\PY{p}{,}
    \PY{l+s+s1}{\PYZsq{}}\PY{l+s+s1}{Asian Put}\PY{l+s+s1}{\PYZsq{}}\PY{p}{:} \PY{n}{asian\PYZus{}puts}\PY{p}{,}
    \PY{l+s+s1}{\PYZsq{}}\PY{l+s+s1}{Lookback Call}\PY{l+s+s1}{\PYZsq{}}\PY{p}{:} \PY{n}{lookback\PYZus{}calls}\PY{p}{,}
    \PY{l+s+s1}{\PYZsq{}}\PY{l+s+s1}{Lookback Put}\PY{l+s+s1}{\PYZsq{}}\PY{p}{:} \PY{n}{lookback\PYZus{}puts}
\PY{p}{\PYZcb{}}
\PY{c+c1}{\PYZsh{} Create a DataFrame}
\PY{n}{df3} \PY{o}{=} \PY{n}{pd}\PY{o}{.}\PY{n}{DataFrame}\PY{p}{(}\PY{n}{data3}\PY{p}{)}\PY{o}{.}\PY{n}{set\PYZus{}index}\PY{p}{(}\PY{l+s+s1}{\PYZsq{}}\PY{l+s+s1}{Strike}\PY{l+s+s1}{\PYZsq{}}\PY{p}{)}

\PY{n}{df3}
\end{Verbatim}
\end{tcolorbox}

            \begin{tcolorbox}[breakable, size=fbox, boxrule=.5pt, pad at break*=1mm, opacityfill=0]
\prompt{Out}{outcolor}{19}{\boxspacing}
\begin{Verbatim}[commandchars=\\\{\}]
        Asian Call  Asian Put  Lookback Call  Lookback Put
Strike
90       12.596374   0.670206      27.812475      4.714733
100       5.760318   3.346444      18.300181     11.715855
110       1.987817   9.086238      10.585264     21.228149
\end{Verbatim}
\end{tcolorbox}
        
    These results closely resemble our earlier observations when we varied
the initial stock price data. We can still draw the same conclusions for
both Asian Options and Lookback Options. The pricing patterns remain
consistent, with Asian Options generally having lower prices compared to
Lookback Options under similar parameters.

    \section{Observations and problems
encountered}\label{observations-and-problems-encountered}

    \textbf{Observations:}

\begin{itemize}
\item
  The volatility of the average stock price over time is lower than the
  volatility of the individual stock. This results in significantly
  lower prices for Asian Options compared to Lookback Options under the
  same parameters. As a result, the reduction in the upfront premium in
  an option contract tends to make Asian Options more appealing to
  investors.
\item
  The price of Lookback Options is significantly higher than that of
  Asian Options. This can be attributed to the extreme payoff structure
  of Lookback Options, which tends to make the contracts more expensive.
\item
  The Risk Neutral Framework makes the pricing of these options
  relatively straightforward, especially when using the Monte Carlo
  Method.
\item
  The Euler-Maruyama Scheme is relatively easy to program, but the
  quality of the entire algorithm depends on the quality of the
  pseudo-random number generator (RNG) methods. To maintain consistent
  results, I have adopted a method that creates a global numpy RNG and
  passes the seed only once. This approach ensures ease of reproduction
  without affecting the randomness of the results \hyperref[5]{[5]}.
\end{itemize}

    \textbf{Problems:}

\begin{itemize}
\item
  In the Asian Options pricing function, I haven't utilized the Average
  Strike payoff and Geometric Average sampling technique.
\item
  In the Lookback Options pricing function, I haven't incorporated the
  Lookback Floating Strike Payoff and Discrete sampling technique.
\item
  The output of the Euler-Maruyama Scheme in (5) depends on the quality
  of the pseudo-random number generator (RNG) for \(\phi\). I only used
  the \texttt{np.random} method and didn't have the time to compare the
  effects of different RNG methods on the output of our Monte Carlo
  model.
\item
  The Euler-Maruyama Scheme has an error of \(O(\delta t)\). Better
  approximations, such as the Milstein method with an error of
  \(O(\delta t^2)\), could be implemented to improve the model.
\end{itemize}

    \section{Conclusion}\label{conclusion}

    Under the risk-neutral framework, the fair value of an option is the
present value of the expected payoff at expiry under a risk-neutral
random walk for the underlying.

\[
V(S,t) = e^{-r(T-t)}\mathbb{E}^\mathbb{Q}[\mathbf{Payoff}(S_{T})]
\]

In this expression, we see the short-term interest rate playing two
distinct roles. First, it is used for discounting the payoff to the
present. This is the term \(e^{-r(T - t)}\) outside the expectation.
Second, the return on the asset in the risk-neutral world is expected to
be \(rS \: dt\) in a time step \(dt\).

I have successfully priced Asian Options and Lookback Options using the
Euler-Maruyama Scheme. The Monte Carlo algorithm is easy to program. The
more simulations I use, the better accuracy I will achieve. The
algorithm is also useful for pricing strong-path-dependent contracts,
such as Asian Options and Lookback Options. The code is easy to
reproduce, and I can create a Python class for later use.

After pricing the options, I noticed that the price of Asian Options is
lower compared to Lookback Options with similar parameters. I also
observed the effect of varying the volatility and strike price on the
option price.

I believe using the Finite Difference Method for pricing these options
should be optimal since it is well-suited for three-dimensional
problems. If I were to do this again, I would like to try the Finite
Difference Method and compare the results between the Monte Carlo
Simulation and Finite Different Method.

    \section{References}\label{references}

    {[}1{]} Module 1 - Lecture 4,Apply Ito \(\text{III}\) for \(V = V(t,X)\)

{[}2{]} Paul Wilmott on Quantitative Finance, Chapter 80, Page 1266

{[}3{]} Paul Wilmott on Quantitative Finance, Chapter 25, Page 428

{[}4{]} Paul Wilmott on Quantitative Finance, Chapter 26, Page 445

{[}5{]} Good practices with numpy random number generators, Albert
Thomas
(https://albertcthomas.github.io/good-practices-random-number-generators/)


    % Add a bibliography block to the postdoc
    
    
    
\end{document}
